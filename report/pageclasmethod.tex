\subsection{Method}
\label{subsec:pageclasmethod}
Histogram of orientated gradients (HOG) features are
to local image descriptors that can be used in image classification and object
detection\cite{dalal2005histograms}. To obtain HOG features from an image, the
image is divided into connected regions called cells. Each cells bins the
orientation of the gradient of the pixels. These cells can then be normalized
over a larger region called blocks to be more robust versus changes in
illumination.

% \begin{frame}
% \frametitle{Features}
% Local features to capture the difference between text and images:
% 	\begin{block}{Histogram of Oriented Gradients (HOG)}
% 		HOG features contain the amount of gradients in a certain image patch.
% 	\end{block}
% 	\begin{block}{Steps for computing HOG Features\cite{dalal2005histograms}}
% 	\begin{enumerate}
% 		\item Global image normalisation
% 		\item Compute the gradient images
% 		\item Compute gradient histograms in 8 directions
% 		\item Normalise across blocks
% 		\item Flatten into a feature vector
% 	\end{enumerate}
% 	\end{block}
% \end{frame}

After calculating the hog features for all pages, one in five of the pages is
selected for the validation set. The rest of the pages are used as training set.
This ensures that all books are represented by both sets. The results in the
next section are found on a freshly downloaded and annotated test set.

\todo{Describe learning the SVM}

% It is my (Maartens) opinion that we should describe our parameters for HOG
% features at the same time as our parameters for the SVMs. Maybe we should even
% move this to the "results" section.
For the classifier, each page was divided into 5 by 5 blocks. Because the pages
are quite alike in terms of lighting conditions \footnote{they are all scanned
similarly, and pages are mostly black on white}, each block had only 1 cell. For
calculating the HOG features, the gradients were binned into eight orientations. 
