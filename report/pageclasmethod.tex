\subsection{Method}
\label{subsec:pageclasmethod}
Histogram of orientated gradients (HOG) features are
to local image descriptors that can be used in image classification and object
detection\cite{dalal2005histograms}. To obtain HOG features from an image, the
image is divided into connected regions called cells. Each cells bins the
orientation of the gradient of the pixels. These cells can then be normalized
over a larger region called blocks to be more robust versus changes in
illumination.

\todo{insert used parameters here}


% \begin{frame}
% \frametitle{Features}
% Local features to capture the difference between text and images:
% 	\begin{block}{Histogram of Oriented Gradients (HOG)}
% 		HOG features contain the amount of gradients in a certain image patch.
% 	\end{block}
% 	\begin{block}{Steps for computing HOG Features\cite{dalal2005histograms}}
% 	\begin{enumerate}
% 		\item Global image normalisation
% 		\item Compute the gradient images
% 		\item Compute gradient histograms in 8 directions
% 		\item Normalise across blocks
% 		\item Flatten into a feature vector
% 	\end{enumerate}
% 	\end{block}
% \end{frame}

\todo{Describe (shortly) the train, validation and test set}

\todo{Describe learning the SVM}

