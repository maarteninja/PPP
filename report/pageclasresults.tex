\subsection{Results}
\label{subsec:pageclasresults}

In order to use the page classification as a preproccessing step for annotating
books, a page classifier would have to have a very high recall. Therefore, the
results shown in this section, are validated using the F2 score on the image
class. 

% It is my (Maartens) opinion that we should describe our parameters for HOG
% features at the same time as our parameters for the SVMs. Maybe we should even
% move this to the "results" section.
Each page was divided into 5 by 5 blocks. Because the pages
are quite alike in terms of lighting conditions\footnote{they are all scanned
similarly, and pages are mostly black on white}, each block had only 1 cell. For
calculating the HOG features, the gradients were binned into eight orientations.
This means the feature vectors for the Linear SVM have $8 \times 5 \times 5 =
200$ dimensions. 
For the SVM constant \emph{C} was validated with $10^c$, with $c = \{1 \dots
6\}$, L2 (squared hinge) loss was used and the dual problem was solved. 

\begin{table}
\centering
\begin{tabular}{l l l l}
\hline
% & \multicolumn{2}{c}{\emph{Pre-trained}} & \multicolumn{2}{c}{\emph{Direct}} \\
  & \textbf{Image} & \textbf{Text} & \textbf{Nothing} \\\hline
\textbf{Precision} & 0.60  & 0.995 & 0.970 \\
\textbf{Recall} & 0.964  & 0.921 & 0.964 \\
\textbf{F2-score} & 0.741 & 0.956 & 0.967 \\\hline
\end{tabular}
\caption{Resulting scores of the page classifier on each of the three annotated
classes.}
\label{tab:pageclasresults}
\end{table}

% Maybe this could just be in the same table as the previous tabular
\begin{table}
\centering
\begin{tabular}{l l l l}
\hline
%& \multicolumn{2}{c}{\emph{Pre-trained}} & \multicolumn{2}{c}{\emph{Direct}} \\
Real\textbackslash Predicted & \textbf{Image} & \textbf{Text} & \textbf{Nothing} \\\hline
\textbf{Image} & 271 & 9 & 1 \\
\textbf{Text} & 169 & 2074 & 9 \\
\textbf{Nothing} & 10 & 2 & 323\\
\hline
\end{tabular}
\caption{Confusion matrix of the page classifier}
\label{tab:pageclascm}
\end{table}

Table \ref{tab:pageclasresults} shows three measures of accuracy of the
classifier on the test set. As can be seen validating using the F2-score
results in a big recall, but in relatively low precision on the image
class as well. These results mean that when this classifier would be used as a
preprocessing step for annotating, about $3.6\%$ of the pages containing images
would not be annotated as such. Table \ref{tab:pageclascm} depicts the confusion
matrix.


