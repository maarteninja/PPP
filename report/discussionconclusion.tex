\section{Discussion \& Future Work}% \& Conclusion}
\label{sec:discussionconclusion}

%\todo{Say we didn't use OCR and maybe why. Also name stroke transform}
We have laid the ground work to segmentate images from text in old books. An
annotator was built which can be extended by implementing a filter
based on the results of the page classifier SVM of section \ref{sec:pageclas}.

Furthermore we have shown that a conditional random field with a 1-slack
structured SVM improves upon the results of a linear SVM that ignores the
neighborhood of its labels.

We have deliberately not chosen to implement existing techniques such as OCR to
recognize text as a feature for image patches, and a sliding window approach to
segmentate images. While this may have hindered performance, we can sure tell
that we have learned more.

For future work the poor performance of the HOG features need to be
investigated. Different parameters in the feature extraction phase, or even
different features all together, might result in more descriptive features.
Additionally, OCR for the features and a sliding window approach could be tried
out. Finally the Hamming loss that is used in solving the structured SVM
could be replaced with a loss that incorporates the rectangle shape of the
bounding boxes for the images.

