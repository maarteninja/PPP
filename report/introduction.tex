\section{Introduction}
\label{sec:introduction}

%\begin{document}
Around the year 2011, Google digitally scanned in a great amount of books from
%\todo{citation needed, where is this information coming from? And why google?
%did not we have a different site as well?}
% No we didn't. That site had only books from the 18th century and we didn't use
% them (they weren't in the list). This info came from
% http://booksearch.blogspot.nl/2011/05/books-from-16th-and-17th-centuries-now.html,
% I don't know how we should cite it if we cite it...
the 16\textsuperscript{th} and 17\textsuperscript{th}
century\cite{bloomburg2011}, also known as the Dutch ``Golden Age''. Many of
these books are now available freely online to read and download. OCR techniques
have been used in order to make some of the book's text available and
searchable. However, the images in the books can only be found by scrolling
through an entire book. Since images are quite sparse in these kind of books,
this can be a tedious task.

Our goal is to extract these images automatically. This could be used to offer
the images separately to users, or make them searchable. Similar to the text
now. We approached the task as a supervised learning problem.

We have divided the this task in several subtasks. In section
\ref{sec:annotator} we first discuss the annotator program we built to quickly
obtain an annotated set of book pages.

As a proof of concept we first do a binary classification of the pages in
section \ref{sec:pageclas}.  Separating the pages with images from those without
them, could result in a speed-up in the annotating process: imagine if one
would only have to specify where the images are on a page, rather than go
through hundreds of pages without any images at all.

Finally, the last step involves the actual problem. We find the bounding boxes of
images in section \ref{sec:imageloc} so that the actual images can be extracted
and presented separately.

However, to get familiar with the problem, the dataset and some of the
challenges that arise with it, are explained, in section \ref{sec:dataset}.

All the code and the produced programs are available publicly at
\url{https://github.com/maarteninja/PPP}

