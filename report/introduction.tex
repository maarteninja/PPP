%\begin{document}
Around the year 2011, Google digitally scanned in a great amount of books from
the 16\textsuperscript{th} and 17\textsuperscript{th} century, also known as the
Dutch ``Golden Age''. Many of these books are now available freely online to
read and download. OCR techniques have been used in order to make each book's
text available and searchable. However, the images in the books can only be
found by scrolling through the entire books. Since images are quite sparse in
these kind of books, this can be a tedious task.

\todo{Meer intro, o.a. Image Segmentation}

\todo{Layout of report with section description}

\section{Dataset}
\todo{More elaborate description (with examples) of books and book pages}

\todo{Add (reference to) statistics of data:}
\begin{table}
\centering
\begin{tabular}{l l}
Total amount of pages: & 5960 \\
Mean pages per book: & 236 \\
Standard deviation: & 223 \\
\hline
Total amount of images: & 525 \\
Mean images per book: & 22.8 \\
Standard deviation: & 49.8
\end{tabular}
\caption{Dataset statistics}
\label{tab:statistics}
\end{table}

