\section{Introduction}
\label{sec:introduction}

%\begin{document}
Around the year 2011, Google digitally scanned in a great amount of books from
\todo{citation needed, where is this information coming from? And why google?
did not we have a different site as well?}
the 16\textsuperscript{th} and 17\textsuperscript{th} century, also known as the
Dutch ``Golden Age''. Many of these books are now available freely online to
read and download. OCR techniques have been used in order to make some of the book's
text available and searchable. However, the images in the books can only be
found by scrolling through an entire book. Since images are quite sparse in
these kind of books, this can be a tedious task.

Our goal is to extract these images automatically. This could be used to offer
the images separately to users, or make them searchable. Similar to the text
now.

We have divided the this task in several subtasks. We approached the task as a
supervised learning problem, so in section \ref{sec:annotator} we first discuss the
annotator program we built to quickly obtain an annotated set of book pages.

As a proof of concept we first classify pages binary in section \ref{pageclas}.
To separate the pages with images from those without could mean a speed up in the
annotating process: imagine if one
would only have to specify where the images are on a page, rather than go
through hundreds of pages without any images at all.

Finally, the last step involves the actual problem we find
the bounding boxes of images in section \ref{sec:imageloc} so that the actual
images can be extracted and presented separately.

However, to get familiar with the problem, the dataset is laid out, and some
challenges that arise with this dataset, in section \ref{sec:dataset}.

All the code and the produces programs are available publicly at
\url{https://github.com/maarteninja/PPP}

