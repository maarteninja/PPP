%\documentclass[draft]{article} % <- compile with draft quickly without images
\documentclass{article}

% TODO: Tweak this margin, I don't know what it should be originally, but we
% need to provide one, in order to be able to change it with \newgeometry (in
% order to fit all the figures on one page)
\usepackage[margin=4cm]{geometry}
%\usepackage{algorithmicx}
%\usepackage{algorithm}
%\usepackage{algpseudocode}
\usepackage{amssymb}
\usepackage{xcolor}
\usepackage{verbatim}
\usepackage{graphicx}
\usepackage{microtype}
\usepackage{caption}
\usepackage{subcaption}
\usepackage{url}
\usepackage{hyperref}
\usepackage{floatrow}
\usepackage{wrapfig}
\usepackage{amsmath}
\usepackage{appendix}


\newfloatcommand{capbtabbox}{table}[][\FBwidth]

\newcommand{\todo}[1]{{\bfseries\sffamily{TODO: #1}}}

\title{Plucking Pictures from Publications\\ \large
Image Segmentation in Old Books}
% Added e-mail addresses because the dataset is available via e-mail (in order
% to prevent legal disputes with Google Books if we put it online somewhere)
\author{Maarten Inja (5872464)\\\small\url{maarten.inja@gmail.com} \and Maarten de
Waard (5894883)\\\small\url{mrtndwrd@gmail.com}}

% Op advies van Thomas:
\date{Intelligent Systems Project, May 2014\\
Supervisor: Thomas Mensink}
\begin{document}

\maketitle

%\begin{figure}[h!]
%\begin{centering}
%\includegraphics{resources/uva}
%\end{centering}
%\end{figure}

\begin{abstract}
In this report we show how we extract images from 16\textsuperscript{th} and
17\textsuperscript{th} century books. These books contain diverse pages and a
new annotator tool was built to create a train and test set. HOG features and a
linear SVM was used to classify pages on whether or not they contain an image or
only text. A conditional random field in combination with a 1-slack structured
SVM  was used to localize and segmentate these images.
\end{abstract}

% explain goal
% explain data set
% show problems data set


\slide{Image Segmentation}
{
% We want to introduce the problem:
%	- Tell about image segmentation
%		Segment image into several classes..?
%	What do we want to tell about image segmentation?
	\begin{columns}
		\begin{column}{.5\textwidth}
			\begin{itemize}
				\item Old books, from 16th and 17th century (Dutch golden age)  are
			scanned and digitally available
				\item For some, the text is available seperately
				\item Goal: extract images to show those seperately as well
			\end{itemize}
		\end{column}
		\begin{column}{.5\textwidth}
			\includegraphics[width=.9\columnwidth]{resources/bookExample}
		\end{column}
	\end{columns}
}

\subsection{Dataset}
\slide{Dataset - Overview 1}
{
	Text pages:
	\begin{columns}
		\begin{column}{.5\textwidth}
			\includegraphics[width=.9\columnwidth]{../data/lhistoireUniverselleDuSieurDavign/raw/500_0043}
		\end{column}
		\begin{column}{.5\textwidth}
			\includegraphics[width=.9\columnwidth]{../data/lesSixVoyagesDeJeanBaptisteTaverni/raw/500_0010}
		\end{column}
	\end{columns}
}
\slide{Dataset - Overview 2}
{
	Image pages:
	\begin{columns}
		\begin{column}{.5\textwidth}
			\includegraphics[width=.9\columnwidth]{../data/naukeurigeBeschryvingVanMoreaEertijt/raw/500_0008}
		\end{column}
		\begin{column}{.5\textwidth}
			\includegraphics[width=.9\columnwidth]{../data/lesSixVoyagesDeJeanBaptisteTaverni/raw/500_0077}
		\end{column}
	\end{columns}
}
\slide{Dataset - Challenge 1}
{
	Difference in quality:
	\includegraphics[width=.8\paperwidth]{resources/example2}
}
\slide{Dataset - Challenge 2}
{
	Pages with useless data:
	\begin{columns}
		\begin{column}{.5\textwidth}
			\includegraphics[width=.9\columnwidth]{../data/staatZugtigeScheepsTogtenEnKrygsBe/raw/500_0002}
		\end{column}
		\begin{column}{.5\textwidth}
			\includegraphics[width=.9\columnwidth]{../data/atlas/raw/500_0004}
		\end{column}
	\end{columns}
}
\slide{Dataset - Challenge 3}
{
	Text and image on the same page
	\begin{columns}
		\begin{column}{.5\textwidth}
			\includegraphics[width=.9\columnwidth]{../data/tweeOngelukkigeScheepsTogtenNaOost/raw/500_0003.png}
		\end{column}
		\begin{column}{.5\textwidth}
			\includegraphics[width=.9\columnwidth]{resources/text_and_image_example}
		\end{column}
	\end{columns}
}

\subsection{Overview}
\slide{Project Overview - Step 1}
{
	\begin{columns}
		\begin{column}{.4\textwidth}
			Step one: annotate
			\begin{enumerate}
				\item Create annotation tool
				\item Classify page as either `text', `useless' or `containing an image'
				\item Annotate bounding boxes of images
			\end{enumerate}
		\end{column}
		\begin{column}{.6\textwidth}
			\includegraphics[width=.8\columnwidth]{resources/screenshot_annotator}
		\end{column}
	\end{columns}
}
\slide{Project Overview - Step 2-3}
{
	Step two: classify pages
	\begin{enumerate}
		\item Calculate 5x5 HOG features per page
		\item Train a Support Vector Machine (SVM) on these feature vectors and
		labels
		\item Predict whether a page contains an image based on this SVM
	\end{enumerate}
	Step three: localize images on pages
	\begin{enumerate}
		\item Calculate 10x20 HOGs per page, these are ``Patches''
		\item Train a Structural Support Vector Machine (SSVM) on a
			Conditional Random Field with these features
		\item Predict per patch if its an image or text patch
	\end{enumerate}
}


\section{Dataset \& Annotator Tool}
\label{sec:dataset}
The dataset consists of books from the 16\textsuperscript{th} and
17\textsuperscript{th} century, which were downloaded from Google Books.
\begin{comment}
%Moved to annotator part of this section
There are several peculiarities. Table
\ref{tab:statistics} shows that the amount of pages per book, and the amount of
images per book differs enormously per book.
\end{comment}
Some examples of the pages can be seen in figure \ref{fig:textImageExamples},
\ref{fig:textExamples}, \ref{fig:imageExamples}, \ref{fig:qualityExamples} and
\ref{fig:baggerExamples}. Figure \ref{fig:qualityExamples} shows that there is
quite a large gap between the quality of scans of some books. The right image of
figure \ref{fig:qualityExamples}
has a vertical line, which indicates that parts of both pages were scanned into the same
image. Furthermore, figure \ref{fig:baggerExamples} shows that some pages
contain something that we can neither classify as text nor as an image.

\begin{table}[h]
\centering
\begin{tabular}{@{\extracolsep{4pt}}l r r @{}}
\hline
 & \textbf{Training set} & \textbf{Test set}\\\hline
\textbf{Total amount of pages:} & 5960 & 2868\\
\textbf{Mean pages per book:} & 236 & 717\\
\textbf{Standard deviation:} & 223 & 290\\
\hline
\textbf{Total amount of images:} & 525 & 286 \\
\textbf{Mean images per book:} & 22.8 & 71.5\\
\textbf{Standard deviation:} & 49.8 & 61.6\\\hline
\end{tabular}
\caption{Dataset statistics}
\label{tab:statistics}
\end{table}

%\section{Annotator}
\label{sec:annotator}

\begin{wrapfigure}[18]{l}{3.5cm}
\vspace{-0.5cm}
\centering
    \includegraphics[width=\textwidth]{resources/screenshotAnnotator}
    \caption{A screen shot of the Annotator tool with an image in a bounding box}
    \label{fig:ann}
\end{wrapfigure}

An annotator was built, in order to quickly and easily annotate all the pages of
a book. When a book is downloaded, the annotator cycles through all its pages.
The user can then press a button on the keyboard indicating whether the page
contains an image, text, or neither. When the page contains one or several
images, the user has to drag a bounding box around the images, using the mouse,
prior to pressing the keyboard button. Errors can be reversed if a page was
skipped too quickly, or if the user felt a bounding box was drawn incorrectly.
Because the majority of the pages are
text pages and the user only has to press 1 button per page, books can be
annotated fairly quickly with this tool.

Using this annotator and the Google Books website, a training set of annotated
book pages was created, containing 5960 pages from 23 books. The statistics of
this dataset can be seen in table \ref{tab:statistics}. A separate test set
containing 2868 pages from 4 books was created as well. Both sets can be
requested via E-mail.





\section{Annotator}
\label{sec:annotator}
quick overview of the annotator goes here


%\begin{document}
% first step: page classification
\begin{frame}
\frametitle{Page Classification}
The first step:
\begin{itemize}
\item Separate the pages containing at least one image from those
containing none
\item Could serve as pre-processing step in annotating
\item Proof of concept
\end{itemize}
\end{frame}


% Briefly explain HOG features

\subsection{Method}

\begin{frame}
\frametitle{Features}
Local features to capture the difference between text and images:
	\begin{block}{Histogram of Oriented Gradients (HOG)}
		HOG features contain the amount of gradients in a certain image patch.
	\end{block}
	\begin{block}{Steps for computing HOG Features\cite{dalal2005histograms}}
	\begin{enumerate}
		\item Global image normalisation
		\item Compute the gradient images
		\item Compute gradient histograms in 8 directions
		\item Normalise across blocks
		\item Flatten into a feature vector
	\end{enumerate}
	\end{block}
\end{frame}


% Explain SVM i.c.w. HOG features for pages
\slide{Classification using SVM}
{
	\begin{itemize}
		\item All pages are annotated with having either ``text'', ``images'' or
		``nothing useful'' on it. Images get bounding boxes, which we will later
		use.
		\item Calculate 5x5 HOG features per page
		\item Train a Support Vector Machine (SVM) on these feature vectors and
		labels
		\item Predict whether a page contains an image based on this SVM
	\end{itemize}
}

\begin{frame}
\frametitle{Test - Validation}
\begin{itemize}
\item Merge the sets of all annoated book pages into one set
\item Split this set into train set ($80\%$) and validation set($20\%$)
\item Use validation set to set parameters ($C$)
\end{itemize}
TODO: total number of pages, mean pages and std. deviation of pages per per book \\
TODO: mean images and std. deviation of images per book
\end{frame}

\subsection{Results}
\begin{frame}
\frametitle{Results}
\begin{itemize}
\item Run the learned classifier on new books
\item Use F2-score in order to focus on recall (preprocess for annotator)
\end{itemize}
TODO: list results

\end{frame}


\todo{Describe Features}

% \begin{frame}
% \frametitle{Features}
% Local features to capture the difference between text and images:
% 	\begin{block}{Histogram of Oriented Gradients (HOG)}
% 		HOG features contain the amount of gradients in a certain image patch.
% 	\end{block}
% 	\begin{block}{Steps for computing HOG Features\cite{dalal2005histograms}}
% 	\begin{enumerate}
% 		\item Global image normalisation
% 		\item Compute the gradient images
% 		\item Compute gradient histograms in 8 directions
% 		\item Normalise across blocks
% 		\item Flatten into a feature vector
% 	\end{enumerate}
% 	\end{block}
% \end{frame}

\todo{Describe (shortly) the train, validation and test set}

\todo{Describe learning the SVM}



\subsection{Results}
\label{subsec:pageclasresults}

In order to use the page classification as a preprocessing step for annotating
books, a page classifier would have to have a very high recall. Therefore, the
results shown in this section, are validated using the F2 score on the image
class. 

% It is my (Waards) opinion that we should describe our parameters for HOG
% features at the same time as our parameters for the SVMs. Maybe we should even
% move this to the "results" section.
Each page was divided into 5 by 5 blocks. Because the pages
are quite alike in terms of lighting conditions\footnote{they are all scanned
similarly, and pages are mostly black on white}, each block had only 1 cell. For
calculating the HOG features, the gradients were binned into eight orientations.
This means the feature vectors for the Linear SVM have $8 \times 5 \times 5 =
200$ dimensions. 
For the SVM constant \emph{C} was validated with $10^c$, with $c = \{1 \dots
6\}$, L2 (squared hinge) loss was used and the dual problem was solved. The
gradient was calculated with a filter size of $1x2$ for each direction.
%\todo{find out filter size (prob. 1x2 filter or something like that). its not
%in the document}

%%% I (Inja) received note that the confusion matrices are flawed and are not to
%%% be included in the report
%%\begin{table}
%%\centering
%%\begin{tabular}{l r r r}
%%\hline
%%%& \multicolumn{2}{c}{\emph{Pre-trained}} & \multicolumn{2}{c}{\emph{Direct}} \\
%%Real\textbackslash Predicted & \textbf{Image} & \textbf{Text} & \textbf{Nothing} \\\hline
%%\textbf{Image} & 271 & 9 & 1 \\
%%\textbf{Text} & 169 & 2074 & 9 \\
%%\textbf{Nothing} & 10 & 2 & 323\\
%%\hline
%%\end{tabular}
%%\caption{Confusion matrix of the page classifier}
%%\label{tab:pageclascm}
%%\end{table}

Table \ref{tab:pageclasresults} shows three measures of accuracy of the
classifier on the test set. As can be seen validating using the F2-score
results in a big recall, but in relatively low precision on the image
class as well. These results mean that when this classifier would be used as a
preprocessing step for annotating, about $3.6\%$ of the pages containing images
would not be annotated as such.
% Table \ref{tab:pageclascm} depicts the confusion matrix.

\begin{table}[h]
\centering
\begin{tabular}{l r r r}
\hline
% & \multicolumn{2}{c}{\emph{Pre-trained}} & \multicolumn{2}{c}{\emph{Direct}} \\
  & \textbf{Image} & \textbf{Text} & \textbf{Nothing} \\\hline
\textbf{Precision} & 0.600  & 0.995 & 0.970 \\
\textbf{Recall}    & 0.964  & 0.921 & 0.964 \\
\textbf{F2-score} l& 0.741  & 0.956 & 0.967 \\\hline
\end{tabular}
\caption{Resulting scores of the page classifier on each of the three annotated
classes.}
\label{tab:pageclasresults}
\end{table}




\begin{frame}
\frametitle{Image Locatization}
Second step:
\begin{itemize}
\item find bounding boxes of images on new books
\item classify whether patch is an image or text patch
\item reconstruct images from patches
\end{itemize}
HOG features:
\begin{itemize}
\item 10x20 per page
\item Additionally: concatenate features, 9x19 per page
\end{itemize}
\end{frame}

\subsection{Method}
\begin{frame}
\frametitle{Conditional Random Fields}
\begin{itemize}
\item Regard image as undirected graph (labels $x_i$, patches $y_i$)
\item Decide label for $x_i$ on patch likeliness, neighborhood, and prior
\item Solver: Structural Support Vector Machine (SSVM)
\end{itemize}
$$ E(\mathbf{x}, \mathbf{y}) = h\sum_i x_i - \beta \sum_{\{i, j\}} x_i x_j
- \eta \sum_i x_i y_i $$

\includegraphics[width=.3\paperwidth]{resources/crf}
\end{frame}

\begin{frame}{CRF and SSVM}

\begin{itemize}
\item To solve the CRF, the energy function must be minimized.
\item For this, two types of SSVMs can be used:
\begin{itemize}
	\item N-slack SSVM
	\item One-slack SSVM
\end{itemize}
\item $\argmin \hat{y} \text{E}(x, y) + \text{loss}(\hat{y}, y) $
\item Where the loss is the \emph{Hamming} loss
\end{itemize}
\end{frame}

\begin{frame}
\frametitle{Preprocess Features Using an SVM}
\begin{itemize}
\item HOG features have 8 values
\item SSVM is harder to solve for more dimensions per feature
\item Use SVM to assign confidence score to each feature
\item Now SSVM has input of 1 dimension per feature
\end{itemize}
\end{frame}

\begin{frame}
\frametitle{Two Stage Training}
Require two stage training to prevent overfitting.
\begin{itemize}
\item Train SVM on $75\%$ of train set, and predict labels on remainder for SSVM
\item Repeat 4 times, with different splits
\item Now $100\%$ of the SSVM features is available
\item Once more: train SVM on $100\%$ of train set to obtain best model
\end{itemize}
\begin{center}
\includegraphics[width=.5\paperwidth]{resources/twostage}
\end{center}
\end{frame}

\subsection{Results}


\begin{frame}
\frametitle{Results, Single Features}

\begin{block}{Scores}
\begin{tabular}{l l l  | l l}
 & \multicolumn{2}{c}{\emph{Pre-trained, overlap}} & \multicolumn{2}{c}{\emph{Direct, overlap}} \\
& \textbf{Image} & \textbf{Text} & \textbf{Image} & \textbf{Text} \\
\textbf{Precision} & ? & ? & ? & ? \\
\textbf{Recall} & ? & ? & ? & ? \\
\textbf{F-score} & ? & ? & ? & ? 
\end{tabular} \\
\end{block}

\begin{block}{Confusion matrices}
\begin{tabular}{l l l | l l }
& \multicolumn{2}{c}{\emph{Pre-trained, overlap}} & \multicolumn{2}{c}{\emph{Direct, overlap}} \\
 & \textbf{Image} & \textbf{Text} & \textbf{Image} & \textbf{Text} \\
\textbf{Image} & ? & ? & ? & ? \\
\textbf{Text} & ? & ? & ? & ?
\end{tabular}
\end{block}
\end{frame}


\begin{frame}
\frametitle{Results, Concatenated Features}

\begin{block}{Scores}
\begin{tabular}{l l l  | l l}
 & \multicolumn{2}{c}{\emph{Pre-trained, overlap}} & \multicolumn{2}{c}{\emph{Direct, overlap}} \\
& \textbf{Image} & \textbf{Text} & \textbf{Image} & \textbf{Text} \\
\textbf{Precision} & 0.269 & 0.979 & ? & ? \\
\textbf{Recall} & 0.743 & 0.854 & ? & ? \\
\textbf{F-score} & 0.395 & 0.912 & ? & ? 
\end{tabular} \\
\end{block}

\begin{block}{Confusion matrices}
\begin{tabular}{l l l | l l }
& \multicolumn{2}{c}{\emph{Pre-trained, overlap}} & \multicolumn{2}{c}{\emph{Direct, overlap}} \\
 & \textbf{Image} & \textbf{Text} & \textbf{Image} & \textbf{Text} \\
\textbf{Image} & 524523 & 58481 & ? & ? \\
\textbf{Text} & 566567 & 5390857 & ? & ?
\end{tabular}
\end{block}



\end{frame}

\todo{Explain CRFs}

\todo{Explain SSVMs}

\todo{Explain different types of processing: preprocessed with SVM and with
either 1 or 4 HOGs per image patch}

\todo{Explain how the results are measured (1 score per image patch)}

\begin{table}
\centering
\begin{tabular}{@{\extracolsep{4pt}}l l l l l @{}}
\hline
 & \multicolumn{2}{c}{\emph{Pre-trained}} & \multicolumn{2}{c}{\emph{Direct}} \\
 \cline{2-3} \cline{4-5}
  & \textbf{Image} & \textbf{Text} & \textbf{Image} & \textbf{Text} \\
\textbf{Precision} & 0.269 & 0.979 & 0.241 & 0.979 \\
\textbf{Recall} & 0.743 & 0.854 &  0.750 & 0.830 \\
\textbf{F-score} & 0.395 & 0.912 & 0.365 & 0.898 \\\hline
\end{tabular}
\caption{scores for image localization}
\label{tab:imagelocresults}
\end{table}

\begin{table}
\centering
% Extra colsep creates fancy lines
\begin{tabular}{@{\extracolsep{4pt}}l l l l l @{}}
\hline
& \multicolumn{2}{c}{\emph{Pre-trained}} & \multicolumn{2}{c}{\emph{Direct}}
\\\cline{2-3}\cline{4-5}
Real\textbackslash Predicted & \textbf{Image} & \textbf{Text} & \textbf{Image} & \textbf{Text} \\
\textbf{Image} & 524523 & 58481 & 524759 & 58245 \\
\textbf{Text} & 566567 & 5390857 & 577927 & 5379497 \\\hline
\end{tabular}
\caption{Confusion Matrix for image localization}
\label{tab:imageloccm}
\end{table}

\todo{Explain results}


DISCUSSION AND CONCLUSION
\todo{Say we didn't use OCR and maybe why. Also name stroke transform}


\bibliographystyle{abbrv}
\bibliography{references}

\newpage
\appendixpage

% Give the images some space!
%\newgeometry{margin=3cm}
\begin{figure}[H]
	\centering
	\begin{subfigure}[b]{0.49\textwidth}
		\includegraphics[width=.49\textwidth]{resources/pageImageExample}
		\includegraphics[width=.49\textwidth]{resources/pageImageExample2}
		\caption{Examples of pages that contain both text and images}
		\label{fig:textImageExamples}
	\end{subfigure}
	\begin{subfigure}[b]{0.49\textwidth}
		\includegraphics[width=.49\textwidth]{resources/500_0043}
		\includegraphics[width=.49\textwidth]{resources/500_0010}
		\caption{Examples of pages that consist of text}
		\label{fig:textExamples}
	\end{subfigure}
	\begin{subfigure}[b]{0.49\textwidth}
		\includegraphics[width=.49\textwidth]{resources/500_0008}
		\includegraphics[width=.49\textwidth]{resources/500_0077}
		\caption{Examples of pages that consist of images}
		\label{fig:imageExamples}
	\end{subfigure}
	\begin{subfigure}[b]{0.49\textwidth}
		\includegraphics[width=.49\textwidth]{resources/good_quality}
		\includegraphics[width=.49\textwidth]{resources/bad_quality}
		\caption{The quality differs per book}
		\label{fig:qualityExamples}
	\end{subfigure}
	\begin{subfigure}[b]{0.49\textwidth}
		\includegraphics[width=.49\textwidth]{resources/500_0002}
		\includegraphics[width=.49\textwidth]{resources/500_0004}
		\caption{Some pages contain neither text nor image}
		\label{fig:baggerExamples}
	\end{subfigure}
	\caption{Examples of book pages from the dataset}
%\todo{Fix the new page that occurs before this figure}}
	\label{fig:examples}
\end{figure}
%\restoregeometry


\end{document}
% vim: set spell :
