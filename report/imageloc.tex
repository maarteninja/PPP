\section{Image Localization}
\label{sec:imageloc}

For the final application of this research, we need only the images, in stead of
the entire page. HOG features have been used for localization, in combination
with a sliding window approach\cite{harzallah2009combining,
suard2006pedestrian}. With this approach, an image is divided into several,
overlapping sub-images. Each window is then classified by
the SVM in much the same way as the Page Classifier works. Such an approach
would not consider the structure of the labels of image
patches\footnote{We did not like this approach because we have implemented this
in the past many times}.

A Conditional Random Field (CRF) \cite{lafferty2001conditional} is an undirected
graphical model which can be used for structured learning. Depending on the
choice of the graph structure, the model allows labels to be conditionally
dependent of neighboring labels. CRFs have been used, with success, to recognize
objects in images\cite{quattoni2004conditional}.

Typically, a CRF has an energy function, and a loss function that maximizes this
energy function. \cite{quattoni2004conditional} introduce hidden variables so
that standard inference and parameter estimation algorithms can be used, i.e.
belief propagation. We have deviated from this approach and use no hidden
variables, but instead train a one-slack structured SVM \cite{joachims2009cutting}.

%\todo{I'm not sure if we want to cite this paper here: artan2010prostate. It
%does seem to be like what we're doing, but it's kind of hard to read and they
%use a cost-sensitive SVM (which apparently is not the same as an SSVM)}


