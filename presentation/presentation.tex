%\documentclass[aspectratio=169]{beamer}
\documentclass{beamer}
\usepackage[utf8x]{inputenc}
%\usepackage{default}
\usepackage{verbatim}
\usepackage{graphicx}
\usepackage{transparent}
\usepackage{listings}
\usepackage{qtree}
\usepackage{amsmath}

% Table of contents weg
% slides nummeren
% introslide over ons algehele probleem al een plaatje laten zien van de boeken
% conclusie data set bekijken: challenges: verschil kwaliteit (in teksts neerzetten)

% Introslide laten zien (met onze 2 stappen, sfm en crfs ssvm), overview methode + annotater tool
%   als 0de stap toevoegen
% page classification en image localization tegelijk laten zien (weer met plaatjes)
% HOG features uitleggen (algemeen HOG feature plaatje, uitleggen hoe je dit
%   plaatje moet laten lezen)
% voorbeeldslides HOG features

% vraag: wat voor schaal voor de HOG hebben we gebruikt (grootte filter)
% HOGgles runnen om te kijken waarom ons project zo faalt

% discussion: focus ook op goede dingen






\DeclareMathOperator*{\argmin}{arg\,min}

\title{Plucking Pictures from Publications}
\subtitle{I don't know what I'm doing!}
\author{Maarten Inja \and Maarten de Waard}
\institute[UvA]{University of Amsterdam}
\date[2014]{Intelligent Systems Project, June 26, 2014}
\logo{\includegraphics[width=45px]{resources/uva}}
\usetheme{Berkeley}
\usecolortheme{sidebartab}
\usefonttheme{structuresmallcapsserif}
\newcommand{\slide}[2]
{
\begin{frame}
\frametitle{#1} 

#2

\end{frame}
}

\begin{document}

% \usebackgroundtemplate{
% {\transparent{0.5}\includegraphics[width=\paperwidth, % height=\paperheight]{resources/dna}}
% }
\begin{frame}
\titlepage
\end{frame}

\slide{Table of Contents}
{
	\tableofcontents
}

\section{Introduction}
% explain goal
% explain data set
% show problems data set


\slide{Image Segmentation}
{
% We want to introduce the problem:
%	- Tell about image segmentation
%		Segment image into several classes..?
%	What do we want to tell about image segmentation?
	\begin{columns}
		\begin{column}{.5\textwidth}
			\begin{itemize}
				\item Old books, from 16th and 17th century (Dutch golden age)  are
			scanned and digitally available
				\item For some, the text is available seperately
				\item Goal: extract images to show those seperately as well
			\end{itemize}
		\end{column}
		\begin{column}{.5\textwidth}
			\includegraphics[width=.9\columnwidth]{resources/bookExample}
		\end{column}
	\end{columns}
}

\subsection{Dataset}
\slide{Dataset - Overview 1}
{
	Text pages:
	\begin{columns}
		\begin{column}{.5\textwidth}
			\includegraphics[width=.9\columnwidth]{../data/lhistoireUniverselleDuSieurDavign/raw/500_0043}
		\end{column}
		\begin{column}{.5\textwidth}
			\includegraphics[width=.9\columnwidth]{../data/lesSixVoyagesDeJeanBaptisteTaverni/raw/500_0010}
		\end{column}
	\end{columns}
}
\slide{Dataset - Overview 2}
{
	Image pages:
	\begin{columns}
		\begin{column}{.5\textwidth}
			\includegraphics[width=.9\columnwidth]{../data/naukeurigeBeschryvingVanMoreaEertijt/raw/500_0008}
		\end{column}
		\begin{column}{.5\textwidth}
			\includegraphics[width=.9\columnwidth]{../data/lesSixVoyagesDeJeanBaptisteTaverni/raw/500_0077}
		\end{column}
	\end{columns}
}
\slide{Dataset - Challenge 1}
{
	Difference in quality:
	\includegraphics[width=.8\paperwidth]{resources/example2}
}
\slide{Dataset - Challenge 2}
{
	Pages with useless data:
	\begin{columns}
		\begin{column}{.5\textwidth}
			\includegraphics[width=.9\columnwidth]{../data/staatZugtigeScheepsTogtenEnKrygsBe/raw/500_0002}
		\end{column}
		\begin{column}{.5\textwidth}
			\includegraphics[width=.9\columnwidth]{../data/atlas/raw/500_0004}
		\end{column}
	\end{columns}
}
\slide{Dataset - Challenge 3}
{
	Text and image on the same page
	\begin{columns}
		\begin{column}{.5\textwidth}
			\includegraphics[width=.9\columnwidth]{../data/tweeOngelukkigeScheepsTogtenNaOost/raw/500_0003.png}
		\end{column}
		\begin{column}{.5\textwidth}
			\includegraphics[width=.9\columnwidth]{resources/text_and_image_example}
		\end{column}
	\end{columns}
}

\subsection{Overview}
\slide{Project Overview - Step 1}
{
	\begin{columns}
		\begin{column}{.4\textwidth}
			Step one: annotate
			\begin{enumerate}
				\item Create annotation tool
				\item Classify page as either `text', `useless' or `containing an image'
				\item Annotate bounding boxes of images
			\end{enumerate}
		\end{column}
		\begin{column}{.6\textwidth}
			\includegraphics[width=.8\columnwidth]{resources/screenshot_annotator}
		\end{column}
	\end{columns}
}
\slide{Project Overview - Step 2-3}
{
	Step two: classify pages
	\begin{enumerate}
		\item Calculate 5x5 HOG features per page
		\item Train a Support Vector Machine (SVM) on these feature vectors and
		labels
		\item Predict whether a page contains an image based on this SVM
	\end{enumerate}
	Step three: localize images on pages
	\begin{enumerate}
		\item Calculate 10x20 HOGs per page, these are ``Patches''
		\item Train a Structural Support Vector Machine (SSVM) on a
			Conditional Random Field with these features
		\item Predict per patch if its an image or text patch
	\end{enumerate}
}


\section{Page Classification}
%\begin{document}
% first step: page classification
\begin{frame}
\frametitle{Page Classification}
The first step:
\begin{itemize}
\item Separate the pages containing at least one image from those
containing none
\item Could serve as pre-processing step in annotating
\item Proof of concept
\end{itemize}
\end{frame}


% Briefly explain HOG features

\subsection{Method}

\begin{frame}
\frametitle{Features}
Local features to capture the difference between text and images:
	\begin{block}{Histogram of Oriented Gradients (HOG)}
		HOG features contain the amount of gradients in a certain image patch.
	\end{block}
	\begin{block}{Steps for computing HOG Features\cite{dalal2005histograms}}
	\begin{enumerate}
		\item Global image normalisation
		\item Compute the gradient images
		\item Compute gradient histograms in 8 directions
		\item Normalise across blocks
		\item Flatten into a feature vector
	\end{enumerate}
	\end{block}
\end{frame}


% Explain SVM i.c.w. HOG features for pages
\slide{Classification using SVM}
{
	\begin{itemize}
		\item All pages are annotated with having either ``text'', ``images'' or
		``nothing useful'' on it. Images get bounding boxes, which we will later
		use.
		\item Calculate 5x5 HOG features per page
		\item Train a Support Vector Machine (SVM) on these feature vectors and
		labels
		\item Predict whether a page contains an image based on this SVM
	\end{itemize}
}

\begin{frame}
\frametitle{Test - Validation}
\begin{itemize}
\item Merge the sets of all annoated book pages into one set
\item Split this set into train set ($80\%$) and validation set($20\%$)
\item Use validation set to set parameters ($C$)
\end{itemize}
TODO: total number of pages, mean pages and std. deviation of pages per per book \\
TODO: mean images and std. deviation of images per book
\end{frame}

\subsection{Results}
\begin{frame}
\frametitle{Results}
\begin{itemize}
\item Run the learned classifier on new books
\item Use F2-score in order to focus on recall (preprocess for annotator)
\end{itemize}
TODO: list results

\end{frame}


\section{Image Localization}

\begin{frame}
\frametitle{Image Locatization}
Second step:
\begin{itemize}
\item find bounding boxes of images on new books
\item classify whether patch is an image or text patch
\item reconstruct images from patches
\end{itemize}
HOG features:
\begin{itemize}
\item 10x20 per page
\item Additionally: concatenate features, 9x19 per page
\end{itemize}
\end{frame}

\subsection{Method}
\begin{frame}
\frametitle{Conditional Random Fields}
\begin{itemize}
\item Regard image as undirected graph (labels $x_i$, patches $y_i$)
\item Decide label for $x_i$ on patch likeliness, neighborhood, and prior
\item Solver: Structural Support Vector Machine (SSVM)
\end{itemize}
$$ E(\mathbf{x}, \mathbf{y}) = h\sum_i x_i - \beta \sum_{\{i, j\}} x_i x_j
- \eta \sum_i x_i y_i $$

\includegraphics[width=.3\paperwidth]{resources/crf}
\end{frame}

\begin{frame}{CRF and SSVM}

\begin{itemize}
\item To solve the CRF, the energy function must be minimized.
\item For this, two types of SSVMs can be used:
\begin{itemize}
	\item N-slack SSVM
	\item One-slack SSVM
\end{itemize}
\item $\argmin \hat{y} \text{E}(x, y) + \text{loss}(\hat{y}, y) $
\item Where the loss is the \emph{Hamming} loss
\end{itemize}
\end{frame}

\begin{frame}
\frametitle{Preprocess Features Using an SVM}
\begin{itemize}
\item HOG features have 8 values
\item SSVM is harder to solve for more dimensions per feature
\item Use SVM to assign confidence score to each feature
\item Now SSVM has input of 1 dimension per feature
\end{itemize}
\end{frame}

\begin{frame}
\frametitle{Two Stage Training}
Require two stage training to prevent overfitting.
\begin{itemize}
\item Train SVM on $75\%$ of train set, and predict labels on remainder for SSVM
\item Repeat 4 times, with different splits
\item Now $100\%$ of the SSVM features is available
\item Once more: train SVM on $100\%$ of train set to obtain best model
\end{itemize}
\begin{center}
\includegraphics[width=.5\paperwidth]{resources/twostage}
\end{center}
\end{frame}

\subsection{Results}


\begin{frame}
\frametitle{Results, Single Features}

\begin{block}{Scores}
\begin{tabular}{l l l  | l l}
 & \multicolumn{2}{c}{\emph{Pre-trained, overlap}} & \multicolumn{2}{c}{\emph{Direct, overlap}} \\
& \textbf{Image} & \textbf{Text} & \textbf{Image} & \textbf{Text} \\
\textbf{Precision} & ? & ? & ? & ? \\
\textbf{Recall} & ? & ? & ? & ? \\
\textbf{F-score} & ? & ? & ? & ? 
\end{tabular} \\
\end{block}

\begin{block}{Confusion matrices}
\begin{tabular}{l l l | l l }
& \multicolumn{2}{c}{\emph{Pre-trained, overlap}} & \multicolumn{2}{c}{\emph{Direct, overlap}} \\
 & \textbf{Image} & \textbf{Text} & \textbf{Image} & \textbf{Text} \\
\textbf{Image} & ? & ? & ? & ? \\
\textbf{Text} & ? & ? & ? & ?
\end{tabular}
\end{block}
\end{frame}


\begin{frame}
\frametitle{Results, Concatenated Features}

\begin{block}{Scores}
\begin{tabular}{l l l  | l l}
 & \multicolumn{2}{c}{\emph{Pre-trained, overlap}} & \multicolumn{2}{c}{\emph{Direct, overlap}} \\
& \textbf{Image} & \textbf{Text} & \textbf{Image} & \textbf{Text} \\
\textbf{Precision} & 0.269 & 0.979 & ? & ? \\
\textbf{Recall} & 0.743 & 0.854 & ? & ? \\
\textbf{F-score} & 0.395 & 0.912 & ? & ? 
\end{tabular} \\
\end{block}

\begin{block}{Confusion matrices}
\begin{tabular}{l l l | l l }
& \multicolumn{2}{c}{\emph{Pre-trained, overlap}} & \multicolumn{2}{c}{\emph{Direct, overlap}} \\
 & \textbf{Image} & \textbf{Text} & \textbf{Image} & \textbf{Text} \\
\textbf{Image} & 524523 & 58481 & ? & ? \\
\textbf{Text} & 566567 & 5390857 & ? & ?
\end{tabular}
\end{block}



\end{frame}



% Show results for classification
% Show results for different usages of SSVM

\section{Demo}
% Show which images are found (and which are not?) in a live demo

% Improvements: 
% - Different kinds of features
% - 

\section{Discussion}

\slide{Discussion}
{
	More work can be done in this area:
	\begin{itemize}
		\item Features: We could use different types of features 
		\item Recognize text (OCR)
		\item Different sizes of patches and overlap
	\end{itemize}
}
\begin{frame}[allowframebreaks]
        \frametitle{References}
        \bibliographystyle{amsalpha}
        \bibliography{references.bib}
\end{frame}

\begin{frame}
Questions?
\end{frame}
\end{document}
